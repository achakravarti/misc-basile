% file desc-ctrl-1987.tex
%
% a LaTeX version of the old paper on
% https://inis.iaea.org/collection/NCLCollectionStore/_Public/19/059/19059867.pdf
\documentclass[11pt,a4paper,svgnames]{article}
\usepackage[T1]{fontenc}
\usepackage[utf8]{inputenc}
\usepackage{alltt}
\usepackage{xcolor}
\usepackage{relsize}
\usepackage{hyperref}

 % see https://tex.stackexchange.com/a/51349/42406
\hypersetup{
  colorlinks   = true, %Colours links instead of ugly boxes
  urlcolor     = NavyBlue, %Colour for external hyperlinks
  linkcolor    = DarkGreen, %Colour of internal links
  citecolor   = DarkMagenta, %Colour of citations
  frenchlinks = true,
}

\input{generated-git-commit}
\begin{document}

\bibliographystyle{abbrvnat}

\date{august, 1987}

\title{Describing Control}

\author{Jean-Marc \textsc{Fouet}\thanks{Jean-Marc Fouet passed away
    around $\approx 2000$; at time of writing was at \emph{Laboratoire de
      Mécanique et Technologie; ENS Cachan / Univ. Paris VI / CNRS;
      94340 Cachan; France}}
\and
Basile \textsc{Starynkevitch}\thanks{current email in 2019:
    \href{mailto:basile@starynkevitch.net}{\texttt{basile@starynkevitch.net}};
    web:
    \href{http://starynkevitch.net/Basile/}{\texttt{starynkevitch.net/Basile/}};
    at time of writing was employed at \emph{Commissariat à l'Energie
      Atomique; IRDI/DEDR/DEMT/SERMA/LETR; CEN Saclay bat 70; 91191
      GIF/YVETTE CEDEX}}}

  \begin{titlepage}
    \maketitle

    \bigskip

    \begin{center}
      \begin{relsize}{-1.5}
        git commit \texttt{\gitcommit}
      \end{relsize}
    \end{center}

  \end{titlepage}

  \begin{center}
    \emph{(short paper submitted to 10\textsuperscript{th} IJCAI)}\\
      {\relsize{-1.5}{(This was originally published as
    \href{https://inis.iaea.org/collection/NCLCollectionStore/\_Public/19/059/19059867.pdf}{\texttt{inis.iaea.org/collection/NCLCollectionStore/\_Public/19/059/19059867.pdf}}
    and retyped with {\LaTeX} in 2019)}}\\
    \textbf{Keywords:} \emph{Metaknowledge} - \emph{Control} - \emph{Heuristics}
  \end{center}

  \begin{abstract}
    Incremental development and maintenance of large systems imply that control
    be clearly separated from knowledge. Finding efficient control for a given
    class of knowledge is itself a matter of expertise, to which knowledge-based
    methods may and should be applied. We present here two attempts at building
    root systems that may later be tuned by knowledge engineers, using the
    semantics of each particular application. These systems are given heuristics
    in a declarative manner, which they use to control the application heuristics.
    Eventually, some heuristics may be used to compile others (or themselves) into
    efficient pieces of programmed code.
  \end{abstract}

  \section{Introduction: Control is needed and cannot be programmed}
  \label{sec:intro}

  ``Logic must be supplemented by some kind of control structure''
  \cite{Simon-Search-1983}. Not only because it would otherwise be too slow and
  space consuming, but also in most cases to avoid iterative or recursive
  explosions. For instance, the rule of example 1 (please refer to the table at
  the end of this paper) will generate an infinity of facts when applied to any
  fact. Control is all the more needed if, instead of dealing with pure logic,
  one considers incomplete and incoherent knowledge bases, rules including
  programmed predicates (for instance ``\texttt{less than}'') and actions (for instance
  ``\texttt{read}'') with non monotonic side effects (see for instance example 2).

  Attempts were made, since Emycin \cite{VanMelle-Emycin-1980}, to draw
  essential, general purpose inference engines. Algorithms have been proposed
  \cite{Forgy-Rete-1982}. But none is really satisfactory: people still write
  and sell new inference engines, for special  purposes, even in propositional
  logic. In the case of full first order logic, instantiating a variable by a
  term gives rise to undecidable problems; in other words, any algorithm is a
  bad algorithm, too heavy for simple cases and too frail for complicated cases.

  To be efficient, it seems obvious that control should be event-driven, that it
  should make the most of properties attached to the objects it handles
  (constants, predicates, and rules), and of its own state (available time and
  space). This involves semantical information (for instance: ``\texttt{evaluating this
  predicate costs 10 units}'') and control heuristics \cite{Lenat-Heuristics-1982}.
  BUT: (1) a program based on heuristics (a huge
  number of yet unknown heuristics) will necessarily have to be often modified;
  (2) giving heuristics in a prescriptive form implies losing information, in
  comparison with a possible descriptive form (see how many rules can be derived
  from example 3); and (3), a program cannot easily modify itself. We are
  therefore developing - each along slightly different principles -
  EUM \cite{Starynkevitch-EUM-1986} and the Gosseyn
  Machine \cite{Fouet-CompRules-1987} to study the possiblity of delcarative
  control; both researches are inspired by the works of D. Lenat and of J.
  Pitrat \cite{Pitrat-Maciste-1985}.

    %% TODO-2
  {\textcolor{red}{\textbf{TODO-2}}}

    %% TODO-3
  {\textcolor{red}{\textbf{TODO-3}}}
  Algorithms have been proposed \cite{Forgy-Rete-1982}.
    %% TODO-4
  {\textcolor{red}{\textbf{TODO-4}}}


    %% TODO-5
  {\textcolor{red}{\textbf{TODO-5}}}
  This involves semantical information
    %% TODO-6
  {\textcolor{red}{\textbf{TODO-6}}}
  and control heuristics \cite{Lenat-Heuristics-1982,Lenat-Eurisko-1983}
    %% TODO-7
  {\textcolor{red}{\textbf{TODO-7}}}

  \section{Local control, global control}
  \label{sec:loc-glob-control}

  Control can be seen as the way to answer the question: ``what should I do
  next?''. Depending on the level at which the question is asked, possible
  answers might be: ``get a value for that slot'', ``see if you can match
  Rule47 with Fact12'', or ``commit suicide''. But control should not only be
  seen as the way to decide when facing a choice: it also means discovering that
  there is a choice, planning so as to have no choice, or deciding not to do
  something even though it was the only possibility.

  For each task, two classes of questions arise: why am I doing this (global
  questions, in relationship with the final goal), and how can I do it (local
  questions). Every time one meets a tree of possible actions, one has to decide
  not only which son of a node to consider first (local), but also whether the
  tree should be developed depth first or breadth first (global). When evaluating
  a conjunction, the local problem of evaluating each predicate must not hide the
  global problem: should the predicates be evaluated one at a time, and in which
  order, or should the whole conjunction be considered. For instance, the first
  strategy will fail if one tries to find all x's satisfying example 5, since any
  of the trhee predicates generates an infinite set. Once it has been globally
  decided to iterate over a set, local methods should decide whether to iterate
  from 1 to n, from n to 1, for all even ranks first, etc (see examples 6, 7 and
  8).

  \section{\textit{A posteriori} control, \textit{a priori} control}
  \label{sec:posteriori-priori-control}

  When an item reaches a fork, the decision to send it in one direction rather
  than the others can be made either by asking for advice or by following a
  pre-imposed plan. The first method is rather easily implemented: each time a
  possible choice occurs, the procedure which should (but doesn't want to) choose
  manufactures a message describing the choice and waits for the reply; the
  inference engine is called recursively, applying rules (like the one of example
  4) to the message.

  On the other hand, the time needed to match the message with the available
  rules may soon become so prohibitive that this method cannot reasonably be
  contemplated, at least on the local levels. Each time knowledge permits, we
  rather like to spend some time analysing the task before launching it; this
  results in hints, associated with each item, and aimed at the procedures
  through which the item should travel. These hints can take a simple form
  (``when building the conflict set, you should only find two possibilities:
  discard the one dealing with P\ldots'') or can really look like a small program.
  Of course, when that method fails (the plan cannot be applied), the first one
  (call for help) is still available.

  \section{Changing representations}
  \label{sec:changing-repr}

  When spying upon the system, one may discover that the same questions are
  asked over and over again, and the same hints are reptitively manufactured at
  great expense. It then seems obvious that one should gradually transform
  knowledge representations so as to diminish the ratio: time spent thinking /
  time spent acting. Like any other optimistation problem, this one involves
  contradictory criteria. ``Caching'', for instance, is useful most of the time,
  but results in an enormous waste of space if used systematically (example 9).

  Some internal representations are better than others in certain circumstances:
  we have no arbitrary way of deciding, for instance, between object oriented and
  relationship oriented paradigms (example 10). For EUM, the ultimate
  representation consists of C procedures, none of which will have been written
  by the author, whereas the Gosseyn Machine aims at the design of its own circuits.
  This very much looks like program specification and control abstraction. In both
  cases, knowledge is still retained in declarative form, to allow for
  ``conscious'' relection when compiled knowlege fails.

  \section{Conclusion: bootstrapping}
  \label{sec:concl-bootstrap}

  We need good heuristics to manufacture the very good heuristics that will take
  their place \cite{Pitrat-Maciste-1985}. We also need a program to use these
  heuristics until they destroy the program. Writing these programs was not very
  inspiring, but we finally came over it, and are presently tuning the first sets
  of heuristics.

  Our representations are highly redundant, including many attributes which are
  not yet useful. Performances are catastrophic, of course. Once they start
  improving, we intend to add dynamic introspection (observing one's reactions)
  to the actual static ability to observe and modify one's structure.

  \section*{Examples}
  \label{sec:examples}

  \begin{enumerate}
  \item if $x$ does not contain \texttt{if-then}, then $x$ is not a rule.
  \item if $x$ is a task, and $x$ is in the agenda, and $x$ has been satisfied,
        then remove $x$ from the agenda.
  \item if $x$ is a rule, and $y$ is a premise of $x$, and $y$ is false, then
        $x$ cannot be fired.
  \item if $x$ is an integer, and $x$ > 5, and $x$ < 10\ldots
  \item if you need an integer, then try 0 and 1 first.
  \item if you are looking for elements of a set $E$, if the cardinality of $E$
        is more than 100, try and reduce $E$ first.
  \item if you are looking for elements of a set $e$, if the cardinality of $E$
        is infinite, never generate more than one element at a time.
  \item if $x$ is a method, if the time spent to find $x$ was more than 10
          units, then file $x$ for future needs.
  \item if $x$ is a functional predicate, and\ldots, then create a slot for $x$.
  \end{enumerate}

\clearpage
\addcontentsline{toc}{section}{References}


\bibliography{bib-desc-ctrl-1987}

\end{document}
%%%%%%%%%%%%%%%%%%%%%%%%%%%%%%%%%%%%%%%%%%%%%%%%%%%%%%%%%%%%%%%%
%% Local Variables: ;;
%% compile-command: " ./build-pdf.sh" ;;
%% End: ;;
%%%%%%%%%%%%%%%%%%%%%%%%%%%%%%%%%%%%%%%%%%%%%%%%%%%%%%%%%%%%%%%%
