% file desc-ctrl-1987.tex
%
% a LaTeX version of the old paper on
% https://inis.iaea.org/collection/NCLCollectionStore/_Public/19/059/19059867.pdf
\documentclass[11pt,a4paper,svgnames]{article}
\usepackage[T1]{fontenc}
\usepackage[utf8]{inputenc}
\usepackage{alltt}
\usepackage{xcolor}
\usepackage{relsize}
\usepackage{hyperref}

 % see https://tex.stackexchange.com/a/51349/42406
\hypersetup{
  colorlinks   = true, %Colours links instead of ugly boxes
  urlcolor     = NavyBlue, %Colour for external hyperlinks
  linkcolor    = DarkGreen, %Colour of internal links
  citecolor   = DarkMagenta, %Colour of citations
  frenchlinks = true,
}

\input{generated-git-commit.tex}
\begin{document}

\bibliographystyle{abbrvnat}

\date{august, 1987}

\title{Describing Control}

\author{Jean-Marc \textsc{Fouet}\thanks{Jean-Marc Fouet passed away
    around $\approx 2000$; at time of writing was at \emph{Laboratoire de
      Mécanique et Technologie; ENS Cachan / Univ. Paris VI / CNRS;
      94340 Cachan; France}}
\and
Basile \textsc{Starynkevitch}\thanks{current email in 2019:
    \href{mailto:basile@starynkevitch.net}{\texttt{basile@starynkevitch.net}};
    web:
    \href{http://starynkevitch.net/Basile/}{\texttt{starynkevitch.net/Basile/}};
    at time of writing was employed at \emph{Commissariat à l'Energie
      Atomique; IRDI/DEDR/DEMT/SERMA/LETR; CEN Saclay bat 70; 91191
      GIF/YVETTE CEDEX}}}

  \begin{titlepage}
    \maketitle
  \end{titlepage}

  \begin{center}
    \emph{(short paper submitted to 10\textsuperscript{th} IJCAI)}\\
      {\relsize{-1.5}{(This was originally published as
    \href{https://inis.iaea.org/collection/NCLCollectionStore/\_Public/19/059/19059867.pdf}{\texttt{inis.iaea.org/collection/NCLCollectionStore/\_Public/19/059/19059867.pdf}}
    and retyped with {\LaTeX} in 2019)}}\\
    \textbf{Keywords:} \emph{Metaknowledge} - \emph{Control} - \emph{Heuristics}
  \end{center}

  \begin{abstract}
    Incremental development and maintenance of large systems imply that control
    be clearly separated from knowledge. Finding efficient control for a given
    class of knowledge is itself a matter of expertise, to which knowledge-based
    methods may and should be applied. We present here two attempts at building
    root systems that may later be tuned by knowledge engineers, using the
    semantics of each particular application. These systems are given heuristics
    in a declarative manner, which they use to control the application heuristics.
    Eventually, some heuristics may be used to compile others (or themselves) into
    efficient pieces of programmed code.
  \end{abstract}

  \section{Introduction: Control is needed and cannot be programmed}
  \label{sec:intro}

  ``Logic must be supplemented by some kind of control structure''
  \cite{Simon-Search-1983}. Not only because it would otherwise be too slow and
  space consuming, but also in most cases to avoid iterative or recursive
  explosions. For instance, the rule of example 1 (please refer to the table at
  the end of this paper) will generate an infinity of facts when applied to any
  fact. Control is all the more needed if, instead of dealing with pure logic,
  one considers incomplete and incoherent knowledge bases, rules including
  programmed predicates (for instance ``less than'') and actions (for instance
  ``read'') with non monotonic side effects (see for instance example 2).

  Attempts were made, since Emycin \cite{VanMelle-Emycin-1980}, to draw
  essential, general purpose inference engines. Algorithms have been proposed
  \cite{Forgy-Rete-1982}. But none is really satisfactory: people still write
  and sell new inference engines, for special  purposes, even in propositional
  logic. In the case of full first order logic, instantiating a variable by a
  term gives rise to undecidable problems; in other words, any algorithm is a
  bad algorithm, too heavy for simple cases and too frail for complicated cases.

  To be efficient, it seems obvious that control should be event-driven, that it
  should make the most of properties attached to the objects it handles
  (constants, predicates, and rules), and of its own state (available time and
  space). This involves semantical information (for instance: ``evaluating this
  predicate costs 10 units'') and control heuristics \cite{Lenat-Heuristics-1982}.
  BUT: (1) a program based on heuristics (a huge
  number of yet unknown heuristics) will necessarily have to be often modified;
  (2) giving heuristics in a prescriptive form implies losing information, in
  comparison with a possible descriptive form (see how many rules can be derived
  from example 3); and (3), a program cannot easily modify itself. We are
  therefore developing - each along slightly different principles -
  EUM \cite{Starynkevitch-EUM-1986} and the Gosseyn
  Machine \cite{Fouet-CompRules-1987} to study the possiblity of delcarative
  control; both researches are inspired by the works of D. Lenat and of J.
  Pitrat \cite{Pitrat-Maciste-1985}.

    %% TODO-2
  {\textcolor{red}{\textbf{TODO-2}}}

    %% TODO-3
  {\textcolor{red}{\textbf{TODO-3}}}
  Algorithms have been proposed \cite{Forgy-Rete-1982}.
    %% TODO-4
  {\textcolor{red}{\textbf{TODO-4}}}


    %% TODO-5
  {\textcolor{red}{\textbf{TODO-5}}}
  This involves semantical information
    %% TODO-6
  {\textcolor{red}{\textbf{TODO-6}}}
  and control heuristics \cite{Lenat-Heuristics-1982,Lenat-Eurisko-1983}
    %% TODO-7
  {\textcolor{red}{\textbf{TODO-7}}}

  \section{Local control, global control}
  \label{sec:loc-glob-control}

  Control can be seen as
    %% TODO-8
  {\textcolor{red}{\textbf{... TODO-8}}}


  \section{\textit{A posteriori} control, \textit{a priori} control}
  \label{sec:posteriori-priori-control}
When an item reaches a fork
    %% TODO-9
  {\textcolor{red}{\textbf{... TODO-9}}}

  \section{Changing representations}
  \label{sec:changing-repr}
  When spying upon the system
    %% TODO-10
  {\textcolor{red}{\textbf{... TODO-10}}}

  \section{Conclusion: bootstrapping}
  \label{sec:concl-bootstrap}
  We need good heuristics
    %% TODO-11
  {\textcolor{red}{\textbf{... TODO-11}}}

  \section*{examples}

  \begin{enumerate}
  \item if $x$ does not contain \texttt{if-then}, then $x$ is not a rule
    \item
    %% TODO-12
      {\textcolor{red}{\textbf{... TODO-12}}}

  \end{enumerate}

\clearpage
\addcontentsline{toc}{section}{References}


\bibliography{bib-desc-ctrl-1987}

\end{document}
%%%%%%%%%%%%%%%%%%%%%%%%%%%%%%%%%%%%%%%%%%%%%%%%%%%%%%%%%%%%%%%%
%% Local Variables: ;;
%% compile-command: " ./build-pdf.sh" ;;
%% End: ;;
%%%%%%%%%%%%%%%%%%%%%%%%%%%%%%%%%%%%%%%%%%%%%%%%%%%%%%%%%%%%%%%%
